% !TeX encoding = UTF-8
\documentclass[14pt,a4paper]{extarticle}

% Настройка русского языка и кодировки
\usepackage[T2A]{fontenc}
\usepackage[utf8]{inputenc}
\usepackage[russian]{babel}

% Настройка полей согласно ГОСТ
\usepackage{geometry}
\geometry{left=30mm,right=15mm,top=20mm,bottom=20mm}

% Настройка абзацев и интервалов
\usepackage{setspace}
\onehalfspacing
\setlength{\parindent}{1.25cm}
\usepackage{indentfirst}

% Настройка для списков
\usepackage{enumitem}
\setlist[enumerate]{label=\arabic*), itemsep=0pt, topsep=5pt, leftmargin=*}
\setlist[itemize]{label={--}, itemsep=0pt, topsep=5pt, leftmargin=*}

% Настройка для таблиц и рисунков
\usepackage{graphicx}
\usepackage{array}
\usepackage{microtype} % Решение проблемы Overfull \hbox

% Гиперссылки
\usepackage{hyperref}
\hypersetup{pdfborder={0 0 0}}

% Команды для заголовков
\newcommand{\headersection}[1]{
  \begin{center}
    \textbf{\MakeUppercase{#1}}
  \end{center}
  \addcontentsline{toc}{section}{#1}
}

\setcounter{page}{7}  % Начать нумерацию с 5

\begin{document}

%%%%%%%% СПИСОК ИСПОЛНИТЕЛЕЙ %%%%%%%%
\headersection{СПИСОК ИСПОЛНИТЕЛЕЙ}

\begin{table}[h]
\begin{tabular}{|c|p{7cm}|p{7cm}|}
\hline
№ & Фамилия, Имя, Отчество & Название и номер раздела \\
\hline
1 & Амурский Василий Андреевич & Реферат, Термины и определения, Перечень сокращений и обозначений, Введение, 1 Анализ предметной области и обоснование необходимости разработки, 2 Проектирование и архитектура системы технического обслуживания воздушных судов, 3 Разработка системы технического обслуживания воздушных судов, 4 Развёртывание и эксплуатация, 5 Итоги разработки и перспективы развития, Заключение, Список использованных источников, Приложения \\
\hline
\end{tabular}
\end{table}

\clearpage

\headersection{РЕФЕРАТ}

Итоговая аттестационная работа состоит из 58 страницы, 3 рисунков, 3 таблиц, 10 использованных источников, 2 приложений.

СИСТЕМА ТЕХНИЧЕСКОГО ОБСЛУЖИВАНИЯ, ВОЗДУШНЫЕ СУДА, МИКРОСЕРВИСНАЯ АРХИТЕКТУРА, ПЛАНИРОВАНИЕ ТЕХНИЧЕСКОГО ОБСЛУЖИВАНИЯ, АВТОМАТИЗАЦИЯ, РАСПРЕДЕЛЕННЫЕ СИСТЕМЫ, ТЕХНОЛОГИЧЕСКИЙ СТЕК, КОНТЕЙНЕРИЗАЦИЯ, DOCKER, KUBERNETES, POSTGRESQL, REDIS, REST API, МОНИТОРИНГ, АНАЛИТИКА

Итоговая аттестационная работа выполнена в формате проекта "Разработка системы технического обслуживания воздушных судов".

Объектом разработки в данной работе является распределенная микросервисная информационная система для автоматизации процессов планирования технического обслуживания, контроля выполнения технических работ, управления запасными частями и формирования аналитической отчетности.

Целью работы является создание отечественного программного решения, адаптированного под требования российской авиационной отрасли, которое обеспечит повышение эффективности обслуживания воздушных судов и снижение зависимости от зарубежных систем.

В процессе работы проведен анализ существующих решений, разработана архитектура системы на основе микросервисного подхода, реализованы ключевые компоненты, спроектированы механизмы хранения и обработки данных, а также организована система развертывания и мониторинга.

Разработанное решение обеспечивает автоматизацию основных бизнес-процессов технического обслуживания воздушных судов, включая управление справочными данными, планирование обслуживания, контроль исполнения работ, управление складскими запасами и формирование аналитической отчетности.

Результаты работы могут быть использованы авиакомпаниями, авиаремонтными предприятиями и сервисными организациями для повышения эффективности процессов технического обслуживания, сокращения времени простоя воздушных судов и оптимизации использования ресурсов.

\clearpage

%%%%%%%% СОДЕРЖАНИЕ %%%%%%%%
\headersection{СОДЕРЖАНИЕ}

\begingroup
\sloppy
\raggedright
\begin{flushleft}

РЕФЕРАТ \dotfill 8

ТЕРМИНЫ И ОПРЕДЕЛЕНИЯ \dotfill 11

ПЕРЕЧЕНЬ СОКРАЩЕНИЙ И ОБОЗНАЧЕНИЙ \dotfill 12

ВВЕДЕНИЕ \dotfill 13

1 АНАЛИЗ ПРЕДМЕТНОЙ ОБЛАСТИ И ОБОСНОВАНИЕ НЕОБХОДИМОСТИ\\
РАЗРАБОТКИ \dotfill 15

\hspace{0.63cm}1.1 Краткое введение в тему проекта и предметную область \dotfill 15

\hspace{0.63cm}1.2 Обоснование выбора темы и формулировка проблемы \dotfill 16

\hspace{0.63cm}1.3 Актуальность разработки \dotfill 16

\hspace{0.63cm}1.4 Veryon \dotfill 16

\hspace{0.63cm}1.5 AMOS (Swiss-AS) \dotfill 17

\hspace{0.63cm}1.6 Skywise (Airbus) \dotfill 18

\hspace{0.63cm}1.7 Итог анализа зарубежных систем \dotfill 18

\hspace{0.63cm}1.8 Выводы по разделу \dotfill 19

2 ПРОЕКТИРОВАНИЕ И АРХИТЕКТУРА СИСТЕМЫ ТЕХНИЧЕСКОГО\\
ОБСЛУЖИВАНИЯ ВОЗДУШНЫХ СУДОВ \dotfill 20

\hspace{0.63cm}2.1 Общее описание структуры системы \dotfill 20

\hspace{0.63cm}2.2 Основные компоненты системы и их функциональные обязанности \dotfill 21

\hspace{0.63cm}2.3 Хранилища данных \dotfill 24

\hspace{0.63cm}2.4 Информационные потоки и взаимодействия \dotfill 25

\hspace{0.63cm}2.5 Архитектурные особенности и технологические решения \dotfill 26

\hspace{0.63cm}2.6 Перспективы развития \dotfill 26

\hspace{0.63cm}2.7 Заключение по разделу \dotfill 27

3 РАЗРАБОТКА СИСТЕМЫ ТЕХНИЧЕСКОГО ОБСЛУЖИВАНИЯ ВОЗДУШНЫХ СУДОВ \dotfill 28

\hspace{0.63cm}3.1 Введение в раздел разработки \dotfill 28

\hspace{0.63cm}3.2 Архитектурные решения \dotfill 29

\hspace{0.63cm}3.3 Описание технологического стека \dotfill 31

\hspace{0.63cm}3.4 Реализация ключевых микросервисов \dotfill 34

\hspace{0.63cm}3.5 Организация хранения и управления данными \dotfill 37

\hspace{0.63cm}3.6 Межсервисное взаимодействие \dotfill 40

\hspace{0.63cm}3.7 Тестирование и обеспечение качества \dotfill 41

4 РАЗВЁРТЫВАНИЕ И ЭКСПЛУАТАЦИЯ \dotfill 44

\hspace{0.63cm}4.1 Контейнеризация и оркестрация (Docker, Kubernetes) \dotfill 44

\hspace{0.63cm}4.2 Мониторинг, логирование и алёртинг \dotfill 46

\hspace{0.63cm}4.3 Особенности безопасности и управления доступом \dotfill 48

5 ИТОГИ РАЗРАБОТКИ И ПЕРСПЕКТИВЫ РАЗВИТИЯ \dotfill 51

\hspace{0.63cm}5.1 Результаты, достижения, выявленные проблемы \dotfill 51

\hspace{0.63cm}5.2 Планы по развитию системы и расширению функционала \dotfill 53

ЗАКЛЮЧЕНИЕ \dotfill 56

СПИСОК ИСПОЛЬЗОВАННЫХ ИСТОЧНИКОВ \dotfill 58

ПРИЛОЖЕНИЕ А \dotfill 60

ПРИЛОЖЕНИЕ Б \dotfill 61
\end{flushleft}
\endgroup
\clearpage

%%%%%%%% РЕФЕРАТ %%%%%%%%
%%%%%%%% РЕФЕРАТ %%%%%%%%


\clearpage

%%%%%%%% ТЕРМИНЫ И ОПРЕДЕЛЕНИЯ %%%%%%%%
\headersection{ТЕРМИНЫ И ОПРЕДЕЛЕНИЯ}

В настоящей итоговой аттестационной работе применяют следующие термины с соответствующими определениями:

Воздушное судно – летательный аппарат, поддерживаемый в атмосфере за счет взаимодействия с воздухом, отличного от взаимодействия с воздухом, отраженным от поверхности земли или воды

Техническое обслуживание воздушных судов – комплекс работ или комплекс операций по поддержанию летной годности воздушного судна при эксплуатации

Микросервисная архитектура – подход к созданию приложения, подразумевающий разбиение на небольшие независимые сервисы, взаимодействующие через API

A-check – форма периодического технического обслуживания легкой степени, проводимая примерно каждые 500-800 летных часов или 2-3 месяца

C-check – форма периодического технического обслуживания тяжелой степени, проводимая примерно каждые 20-24 месяца или 6000-8000 летных часов

D-check – наиболее комплексная форма технического обслуживания воздушного судна, проводимая примерно каждые 6-10 лет

API Gateway – компонент архитектуры, являющийся единой точкой входа для всех клиентов и обеспечивающий маршрутизацию запросов к нужным микросервисам

Контейнеризация – технология виртуализации на уровне операционной системы, позволяющая упаковывать приложение со всеми его зависимостями в изолированный контейнер

Оркестрация – автоматизированное размещение, координация и управление сложными компьютерными системами и службами

SLA (Service Level Agreement) – соглашение об уровне предоставления услуги, формальный договор между заказчиком услуги и её поставщиком, содержащий описание услуги и принципы её предоставления

\clearpage

\clearpage

%%%%%%%% ПЕРЕЧЕНЬ СОКРАЩЕНИЙ И ОБОЗНАЧЕНИЙ %%%%%%%%
\headersection{ПЕРЕЧЕНЬ СОКРАЩЕНИЙ И ОБОЗНАЧЕНИЙ}

В настоящей итоговой аттестационной работе применяют следующие сокращения и обозначения:

ВС – воздушное судно

ТО – техническое обслуживание

ТОиР – техническое обслуживание и ремонт

АТ – авиационная техника

СУБД – система управления базами данных

API – Application Programming Interface (программный интерфейс приложения)

REST – Representational State Transfer (передача состояния представления)

MRO – Maintenance, Repair, and Operations (техническое обслуживание, ремонт и эксплуатация)

CI/CD – Continuous Integration/Continuous Delivery (непрерывная интеграция и доставка)

RBAC – Role-Based Access Control (контроль доступа на основе ролей)

JWT – JSON Web Token (веб-токен в формате JSON)

ELK – Elasticsearch, Logstash, Kibana (стек технологий для поиска, анализа и визуализации данных)

IDS/IPS – Intrusion Detection System/Intrusion Prevention System (система обнаружения/предотвращения вторжений)

WAF – Web Application Firewall (межсетевой экран для веб-приложений)

ERP – Enterprise Resource Planning (планирование ресурсов предприятия)

\clearpage

\clearpage

%%%%%%%% ВВЕДЕНИЕ %%%%%%%%
\headersection{ВВЕДЕНИЕ}

Техническое обслуживание воздушных судов является критически важным аспектом обеспечения безопасности полетов и эффективности эксплуатации авиационной техники. Авиационная отрасль предъявляет особенно высокие требования к качеству, своевременности и документированию всех технических работ, проводимых на воздушных судах.

С увеличением количества воздушных судов, усложнением их конструкции и ужесточением требований к безопасности полетов, традиционные методы планирования и контроля технического обслуживания становятся недостаточно эффективными. Авиакомпании и сервисные организации нуждаются в современных информационных системах, способных автоматизировать и оптимизировать все аспекты процесса технического обслуживания.

В настоящее время на мировом рынке представлены различные системы управления техническим обслуживанием воздушных судов, такие как AMOS, Veryon и Skywise. Однако, как показывает практика, эти решения имеют ряд ограничений для российских авиакомпаний: высокая стоимость приобретения и поддержки, сложность адаптации к национальным стандартам и регламентам, зависимость от зарубежных поставщиков и потенциальные риски прекращения технической поддержки.

В связи с этим актуальной задачей является разработка отечественного программного обеспечения для автоматизации процессов технического обслуживания воздушных судов, учитывающего специфику российской авиационной отрасли и современные технологические тенденции.

Целью данной работы является создание системы технического обслуживания воздушных судов на основе микросервисной архитектуры, обеспечивающей полный цикл управления техническим обслуживанием от планирования до формирования отчетности, с возможностью гибкой адаптации под требования различных авиационных предприятий.

Для достижения поставленной цели необходимо решить следующие задачи:

1) провести анализ существующих аналогов и выявить их ограничения;

2) разработать архитектуру и структурную модель системы, учитывающую специфику предметной области;

3) спроектировать и реализовать ключевые компоненты системы, включая подсистемы управления справочными данными, планирования, исполнения работ, управления складом и формирования аналитики;

4) организовать эффективную систему хранения и обработки данных;

5) разработать механизмы межсервисного взаимодействия;

6) обеспечить надежное и безопасное функционирование системы;

7) создать инфраструктуру для развертывания и мониторинга;

8) определить перспективы дальнейшего развития системы.

Практическая значимость работы заключается в создании готового к внедрению программного решения, способного повысить эффективность процессов технического обслуживания воздушных судов, сократить время простоя авиационной техники и оптимизировать использование ресурсов авиационных предприятий.

\clearpage

\clearpage

%%%%%%%% РАЗДЕЛ 1 %%%%%%%%
\section{АНАЛИЗ ПРЕДМЕТНОЙ ОБЛАСТИ И ОБОСНОВАНИЕ НЕОБХОДИМОСТИ РАЗРАБОТКИ}

\subsection{Краткое введение в тему проекта и предметную область}

Техническое обслуживание воздушных судов представляет собой комплексный и многофункциональный процесс, включающий управление справочными данными, планирование обслуживания, контроль и исполнение заданий, учет запасных частей и ресурсов, а также формирование аналитики и отчетности. Эффективность этих процессов напрямую влияет на безопасность, экономичность и доступность воздушного флота. Автоматизация и цифровизация данных процессов является важнейшей задачей для обеспечения конкурентоспособности авиакомпаний и сервисных предприятий.

Техническое обслуживание воздушных судов структурировано по типам проверок, традиционно обозначаемым как A, B, C, D checks, каждый из которых имеет свою периодичность и объем работ [1]. Учет и планирование этих типов проверок является базовой функцией разрабатываемой системы.

В частности, предметная область охватывает следующие бизнес-процессы:

1) управление основными справочными данными о воздушных судах, персонале, запчастях и регламентах,

2) формирование долговременных и краткосрочных графиков технического обслуживания на основе регламентов и данных о состоянии самолетов,

3) организация и контроль исполнения работ техническими специалистами с учетом квалификации и назначенных заданий,

4) управление складскими запасами, учет остатков и формирование заявок на закупку,

5) генерация регулярной аналитики и отчетов для различных уровней управления и контроля.

\subsection{Обоснование выбора темы и формулировка проблемы}

Разработка интегрированной автоматизированной системы технического обслуживания воздушных судов является актуальной в связи с ростом требований авиационной безопасности, усложнением нормативной базы, а также увеличением объёмов и быстродействия данных, с которыми должны работать авиационные организации. Несмотря на существование зарубежных решений, их адаптация под российские условия осложнена высокой стоимостью лицензирования, особенностями регламентов и локальных систем.

Основная проблема, решаемая в рамках проекта, – отсутствие доступного, масштабируемого и адаптируемого решения, позволяющего объединить ключевые процессы ТОиР в единую платформу с возможностью гибкой интеграции внешних данных и модулей.

\subsection{Актуальность разработки}

Современные информационные системы технологического обслуживания радиоэлектронного оборудования и авиационной техники зарубежных разработчиков характеризуются высокой функциональностью и применяются многими мировыми авиакомпаниями и сервисными центрами. Ниже представлен детальный анализ основных зарубежных конкурентов и аналогичных решений, позволяющий выделить сильные стороны существующих систем, выявить их ограничения и мотивы для разработки отечественного программного продукта.

\subsection{Veryon}

Veryon – облачное решение, ориентированное на управление техническим обслуживанием и ремонтами малых и средних предприятий авиационной сферы. Основные характеристики Veryon включают: 

1) интеграцию с популярными корпоративными системами: SAP, Power BI, Oracle, что позволяет реализовать обширную аналитику и отчетность,

2) поддержку мобильных приложений, предоставляющих доступ к задачам и данным на любом устройстве,

3) интуитивно понятный интерфейс, поддерживающий гибкую настройку бизнес-процессов,

4) наличие встроенных инструментов анализа производительности ремонтов и планирования ресурсов,

5) абонентскую стоимость порядка 1000 долларов США в месяц, что делает систему привлекательной для малого бизнеса, но может быть слишком дорогостоящей для многих российских предприятий.

Однако, несмотря на сильную техническую базу, Veryon имеет ограничения, связанные с недостаточной локализацией под российские авиастандарты и регламенты, а также с высокими затратами на лицензирование.

\subsection{AMOS (Swiss-AS)}

AMOS [8] представляет собой комплексную систему управления техническим обслуживанием и эксплуатацией, широко применяемую крупными авиакомпаниями и МРО-сервисами. Ключевые особенности AMOS включают:

1) полный охват MRO процессов: от планирования обслуживающих работ до складского учета, управления закупками и финансового контроля,

2) модульную структуру, позволяющую адаптировать систему под конкретные требования заказчика и менять функциональность без глобальных изменений архитектуры,

3) интеграцию с внешними системами через API и AMOScentral, обеспечивающую обмен данными с системами управления полетами, складами и логистикой,

4) поддержку мобильных устройств через AMOSmobile, что предоставляет техникам и супервайзерам доступ к заданиям и отчетам на планшетах и смартфонах,

5) высокую степень соответствия международным отраслевым стандартам и регламентам, включая требования EASA и FAA,

6) по отзывам пользователей, высокую сложность настройки и значительные капитальные и операционные затраты на внедрение и сопровождение.

Данные характеристики делают AMOS ориентиром для крупных авиакомпаний, но ограничивают его применимость для организаций, которым требуются гибкие, адаптируемые и менее ресурсоемкие решения.

\subsection{Skywise (Airbus)}

Skywise [9] — это инновационная платформа предиктивного анализа и управления техническим обслуживанием, разработанная Airbus. Система обладает следующими ключевыми особенностями: 

1) основной акцент на использовании больших данных и аналитики для прогнозирования технических проблем и оптимизации планов обслуживания,

2) централизованное хранилище эксплуатационных данных, позволяющее авиакомпаниям получать оперативную информацию о состоянии своего парка,

3) модульная интеграция, адаптирующая функциональность под нужды различных пользователей — от планировщиков до технических специалистов и руководства,

4) использование облачных технологий, обеспечивающих масштабируемость и простоту интеграций,

5) функция мониторинга полётов в реальном времени с уведомлениями о возможных неисправностях, сокращающая время реагирования.

Несмотря на передовые технологические решения, Skywise требует наличия развитой IT-инфраструктуры и высокой квалификации персонала, а также может оказаться недостаточно кастомизируемой под специфику российских условий.

\subsection{Итог анализа зарубежных систем}

Анализ существующих иностранных продуктов показывает, что они обладают мощной функциональностью и технологическими преимуществами, но при этом имеют ряд важных ограничений для российского рынка:

1) высокая стоимость лицензирования и поддержки,

2) ограниченная локализация под национальные стандарты и регламенты,

3) сложность и высокая стоимость внедрения,

4) требования к инфраструктуре и квалификации персонала.

Названные обстоятельства формируют необходимость разработки отечественной системы планирования технического обслуживания воздушных судов, ориентированной на специфику российского рынка и способной обеспечить:

1) гибкую, масштабируемую архитектуру на базе микросервисов,

2) высокую производительность при доступной стоимости,

3) поддержку российских нормативных документов и практик,

4) удобный интерфейс и инструменты для эффективного управления,

5) возможность интеграции с существующими корпоративными системами.

\subsection{Выводы по разделу}

Проведённый анализ международных аналогов и требований отечественного авиационного рынка подтверждает стратегическую важность создания собственной распределённой микросервисной системы планирования технического обслуживания. Разработка будет способствовать повышению уровня безопасности и эффективности эксплуатации воздушных судов, сокращению затрат и оптимизации бизнес-процессов российских авиакомпаний и сервисных предприятий.

Итогом работы станет создание архитектурного решения и проектной документации, служащих базой для разработки и последующего внедрения комплексной системы, полностью отвечающей потребностям отечественной авиационной отрасли.

\newpage


%%%%%%%% РАЗДЕЛ 2 %%%%%%%%
\section{ПРОЕКТИРОВАНИЕ И АРХИТЕКТУРА СИСТЕМЫ ТЕХНИЧЕСКОГО ОБСЛУЖИВАНИЯ ВОЗДУШНЫХ СУДОВ}

\subsection{Общее описание структуры системы}

Проектируемая система технического обслуживания воздушных судов является комплексным информационным решением, призванным обеспечить качество, безопасность и своевременность проведения всех видов технических работ в авиационной отрасли. В основе проектирования лежит модульный принцип, обеспечивающий чёткое разделение функциональных зон и взаимодействие через стандартизированные интерфейсы. Такая архитектура позволяет реализовать гибкую, масштабируемую и легко поддерживаемую платформу.

На рисунке 1 представлена общая структура системы с указанием основных компонентов и их взаимосвязей.

\begin{figure}[h]
\centering
\caption{Общая структура системы технического обслуживания воздушных судов}
\end{figure}

Подсистема планирования обеспечивает автоматизированное формирование графиков для всех типов технического обслуживания в соответствии с международной классификацией проверок A, B, C и D [1], что позволяет поддерживать единые стандарты обслуживания.

В процессе построения архитектуры были выделены ключевые компоненты, представляющие собой отдельные подсистемы, которые взаимодействуют между собой посредством обмена данными и событий. Важной составляющей является организация централизованных хранилищ информации, отражающих актуальное состояние воздушных судов, регламентов, персонала, запасных частей и истории выполненных работ.  

Данная структура обеспечивает устранение дублирования данных, повышает точность планирования и исполнения работ, а также позволяет проводить глубокий анализ и формировать аналитическую отчетность для различных уровней управления и контроля.

\subsection{Основные компоненты системы и их функциональные обязанности}

Система состоит из пяти основных подсистем, каждая из которых отвечает за определенный функциональный блок. В таблице 1 представлено краткое описание назначения каждой подсистемы.

\begin{table}[h]
\caption{Основные подсистемы и их назначение}
\begin{tabular}{|p{1.5cm}|p{4cm}|p{9cm}|}
\hline
Код & Название & Основное назначение \\
\hline
P1 & Подсистема управления справочными данными и интеграции & Централизованное управление справочной информацией и интеграция с внешними системами \\
\hline
P2 & Подсистема планирования технического обслуживания & Формирование и оптимизация планов ТО на основе регламентов и состояния ВС \\
\hline
P3 & Подсистема управления исполнением работ & Распределение заданий и контроль выполнения работ \\
\hline
P4 & Подсистема управления складом и потребностями & Учет запасных частей, формирование заявок на закупку \\
\hline
P5 & Подсистема формирования аналитики и отчётности & Агрегация данных и подготовка отчетов \\
\hline
\end{tabular}
\end{table}

Далее рассмотрим подробнее функциональные обязанности каждой подсистемы.

\subsubsection{Подсистема управления справочными данными и интеграции (P1)}

Данная подсистема является централизованным источником достоверной информации, обеспечивающим поступление и обновление данных из внешних систем. К таким системам относятся: 

1) кадровые мастер-системы, предоставляющие сведения о квалификации и ролевой структуре персонала,

2) складские системы, отражающие наличие и движение запасных частей,

3) телеметрические системы, предоставляющие сведения о состоянии воздушных судов в реальном времени,

4) системы каталогов и производителей, включающие нормативные документы, регламенты и бюллетени.

Особое внимание в подсистеме уделено обеспечению высокого качества данных — реализованы процедуры валидации, очистки и стандартизации. Посредством триггеров и событий осуществляется оповещение остальных подсистем о необходимости актуализации и корректировки планов и заданий.

\subsubsection{Подсистема планирования технического обслуживания (P2)}

Подсистема P2 отвечает за формирование сбалансированных и эффективных графиков технического обслуживания, ориентированных на два основных аспекта: нормативно-правовые требования и состояние воздушных судов. Основной функционал включает: 

1) обеспечение автоматизированной генерации долговременных (годовых, квартальных) планов технического обслуживания,

2) формирование краткосрочных (месячных, недельных) расписаний, оперативно корректируемых на основе показателей телеметрии,

3) реализацию механизмов приоритетного внесения внеплановых действий, связанных с обнаруженными дефектами или неисправностями,

4) интеграцию с Центром управления полётами для синхронизации графиков технического обслуживания с расписанием эксплуатации воздушных судов.

В подсистеме реализованы алгоритмы оптимизации загрузки технического персонала и ресурсов, а также контроля соблюдения регламентов. Ведение версий планов и историзация изменений обеспечивают прозрачность процесса и возможность аудита.

\subsubsection{Подсистема управления исполнением работ (P3)}

П3 является оперативным ядром системы, обеспечивая прямое взаимодействие с техническим персоналом и контролёрами. В задачи подсистемы входит: 

1) распределение заданий согласно планам ТО, с учётом квалификационных требований и загрузки исполнителей,

2) сбор информации о ходе и результатах выполнения технических операций, выявленных дефектах и использовании материалов,

3) интеграция с электронными формами бюллетеней и актов, обеспечивающая полноту и достоверность документации,

4) поддержка механизмов обратной связи и уведомлений пользователей, создание отчётов о выполнении в режиме реального времени.

Данная подсистема обеспечивает постоянный контроль над прогрессом технического обслуживания и способствует снижению количества ошибок и пропусков при плановых и внеплановых работах.

\subsubsection{Подсистема управления складом и потребностями (P4)}

Поддерживающая процессы снабжения и учета запчастей подсистема выполняет следующие функции: 

1) мониторинг запасов на складах с обновлением информации в режиме реального времени,

2) резервирование и списание запчастей на основании заявок и результатов работы технического персонала,

3) автоматическое формирование заявок на закупки при снижении остатков ниже установленных порогов,

4) интеграция с отделами закупок и внешними поставщиками для обеспечения бесперебойного снабжения,

5) создание аналитических отчётов по движению складских запасов, выявлению филиалов с дефицитом и оптимизации процессов закупок.

Подсистема построена с использованием гибких настроек и разграничения прав доступа, что гарантирует высокую безопасность и точность учета.

\subsubsection{Подсистема формирования аналитики и отчётности (P5)}

P5 обеспечивает сбор и комплексный анализ данных системы, формируя необходимые отчёты для внутреннего и внешнего использования: 

1) генерация регулярных сводок о состоянии парка воздушных судов, выполнении планов ТО, выявленных дефектах,

2) подготовка отчетов по использованию запасных частей и расходам,

3) предоставление информации для контролирующих органов в соответствии с нормативными требованиями,

4) разработка интерфейсов и инструментов для визуализации ключевых показателей и прогнозирования.

Функции подсистемы позволяют существенно повысить качество управленческих решений и обеспечить прозрачность деятельности предприятия.

\subsection{Хранилища данных}

Система использует пять основных хранилищ данных, каждое из которых специализируется на определенном типе информации. Эти хранилища обеспечивают централизованное хранение и обработку всех данных, необходимых для функционирования системы.

\subsubsection{DS1 — База данных воздушных судов и их состояния}

Содержит детализированную техническую информацию о воздушных судах, в том числе идентификаторы, технические параметры, историю эксплуатации и показатели телеметрии. Обеспечивает источник достоверных данных для планирования и выполнения ТО.

\subsubsection{DS2 — База данных запчастей и каталогов}

Включает полный перечень компонентов с уникальными идентификаторами, спецификациями и сведениями о совместимости с различными моделями и конфигурациями воздушных судов. Хранит информацию о текущих остатках и движениях запасных частей.

\subsubsection{DS3 — База данных персонала}

Формирует реестр сотрудников технических служб с указанием квалификации, специальностей, прав и ограничений. Используется для назначения задач и контроля за выполнением работ.

\subsubsection{DS4 — База данных планов и регламентов обслуживания}

Хранит нормативные документы и регламенты (РЛ), специфику периодичности технического обслуживания и сформированные графики работ по каждому воздушному судну.

\subsubsection{DS5 — База данных записей о выполненных работах}

Систематизирует всю информацию по выполненным операциям, включая дату выполнения, персонал, использованные запчасти, выявленные дефекты и результаты контроля качества.

\subsection{Информационные потоки и взаимодействия}

Все подсистемы связаны продуманными информационными потоками, направленными на обеспечение согласованности, полноты и своевременности обмена данными. Основные каналы обмена включают:

1) сообщения и данные из внешних систем передаются в подсистему управления справочными данными и интеграции (P1),

2) данные состояния воздушных судов и регламенты используются подсистемой планирования (P2) для формирования графиков ТО,

3) плановые задания и корректировки передаются в подсистему управления исполнением работ (P3),

4) потребности в материалах и запасных частях идут в подсистему управления складом и потребностями (P4),

5) агрегированные и аналитические данные передаются в подсистему формирования отчётности (P5), с которой взаимодействуют управляющие и контролирующие органы.

Система предусматривает двунаправленный обмен данными с внешними пользователями и системами, в том числе техниками, операторами склада, отделом закупок, Центром управления полётами и надзорными учреждениями.

\subsection{Архитектурные особенности и технологические решения}

Предлагаемая архитектура основана на построении модульной распределённой системы с использованием микросервисного подхода. Каждый подсистемный компонент реализуется отдельным сервисом с чётко определённым интерфейсом взаимодействия, что обеспечивает:

1) независимость разработки, тестирования и развёртывания,

2) лёгкость масштабирования и обновления,

3) надёжность и отказоустойчивость системы за счёт распределения ответственности и сдублированных компонентов,

4) возможность интеграции с внешними информационными системами и адаптации под особенности эксплуатации.

Предполагается использование стандартов взаимодействия по RESTful API с передачей данных в формате JSON. Для хранения данных выбираются реляционные и документные базы данных в зависимости от специфики и скорости доступа.

Безопасность и разграничение доступа реализуются средствами аутентификации и авторизации в соответствии с нормативами авиационной отрасли.

Использование микросервисной архитектуры [4] обеспечивает возможность постепенного наращивания функциональности системы и гибкой адаптации к изменяющимся требованиям бизнеса без необходимости полной переработки кодовой базы.

\subsection{Перспективы развития}

Данная архитектура позволяет поэтапно внедрять дополнительные функциональные модули, расширять перечень аналитических инструментов, а также интегрировать сторонние решения, поддерживающие цифровую трансформацию процессов технического обслуживания воздушных судов.

\subsection{Заключение по разделу}

Разработанная архитектура и структурная модель системы отвечают современным требованиям надежности, гибкости и эффективности технического обслуживания воздушных судов. Использование модульного подхода и продуманной организации потоков данных позволит повысить качество планирования, контроля и анализа, что ведет к повышению безопасности и экономической эффективности эксплуатации авиационного парка.

\newpage
% И так далее для разделов 3, 4, 5...


%%%%%%%% РАЗДЕЛ 3 %%%%%%%%
\section{РАЗРАБОТКА СИСТЕМЫ ТЕХНИЧЕСКОГО ОБСЛУЖИВАНИЯ ВОЗДУШНЫХ СУДОВ}

\subsection{Введение в раздел разработки}

Разработка современной информационной системы технического обслуживания воздушных судов представляет собой комплексный процесс, направленный на создание надёжной, масштабируемой и адаптивной платформы, способной обеспечить полную автоматизацию и координацию ключевых процессов ТОиР (техническое обслуживание и ремонт). В условиях высокой конкуренции в авиационной отрасли, усиления требований к безопасности и эффективности эксплуатации воздушных судов возникает необходимость внедрения цифровых решений, обеспечивающих прозрачность и оперативность управления техническими ресурсами.

Цель данного раздела — детально описать этапы проектирования и реализации системы, отражающие требования к надежности, производительности и соответствию международным и национальным стандартам. В процессе разработки учитывалась специфика авиационной отрасли России и стран СНГ, особенности нормативно-правовой базы, а также современный опыт компаний-лидеров отрасли.

В работе решаются следующие ключевые задачи:

\begin{enumerate}
\item Анализ и выбор архитектурной модели, отвечающей разнообразным требованиям функциональности, масштабируемости и безопасности системы,
\item Обоснование выбора программных технологий, включая языки программирования, субсистемы хранения данных и инструменты интеграции,
\item Детальное проектирование и реализация ключевых компонентов системы, обеспечивающих выполнение функциональных и нефункциональных требований,
\item Организация взаимодействия микросервисов через стандартизованные протоколы и асинхронные обмены сообщениями,
\item Организация процесса тестирования с применением современных методик для обеспечения качества и надежности программного продукта,
\item Подготовка технической и эксплуатационной документации, а также организация процессов развёртывания и сопровождения системы в виртуализированных и облачных средах.
\end{enumerate}

Данный раздел подробно раскрывает методики и инструменты, используемые при создании системы. Особое внимание уделено выстраиванию архитектуры, обеспечивающей возможность быстрого реагирования на изменения внешних условий, а также простоты модификации и расширения функциональности согласно требованиям бизнеса.

Выполнение задач на данном этапе гарантирует создание программного комплекса, который станет основой для повышения эффективности работы технических служб, снижая риски простоев и обеспечивая выполнения регламентов с высокой степенью автоматизации.

\subsection{Архитектурные решения}

Выбор архитектурной основы для системы технического обслуживания воздушных судов имел решающее значение для обеспечения её устойчивости, масштабируемости и адаптируемости. После рассмотрения различных подходов, в качестве базовой архитектурной модели была выбрана микросервисная архитектура, которая на сегодняшний день считается оптимальной для построения масштабируемых распределённых систем с высокой степенью автономности компонентов.

Основные преимущества данной архитектуры включают:

\begin{enumerate}
\item Модульность: Архитектура разбивает систему на независимые сервисы, каждый из которых отвечает за ограниченную часть функциональности. Это позволяет проводить разработку, тестирование и развёртывание компонентов независимо друг от друга,
\item Масштабируемость: Отдельные микросервисы могут масштабироваться горизонтально по мере роста нагрузки, что обеспечивает высокую производительность и устойчивость под нагрузкой,
\item Отказоустойчивость: В случае сбоев в одном из сервисов остальные продолжают функционировать, что повышает общую надежность системы.
\end{enumerate}

Архитектурное деление системы было сделано в соответствии с функциональными зонами, описанными на стадии анализа и визуализированными в структурной модели (см. раздел~2):

\begin{enumerate}
\item Сервис управления справочными данными и интеграции (сервис P1) — реализует соединение с внешними источниками информации, обеспечивает механизм загрузки, валидации и унификации данных,
\item Сервис планирования технического обслуживания (сервис P2) — ответственен за построение, оптимизацию и актуализацию планов ТО и ремонта,
\item Сервис управления исполнением работ (сервис P3) — организует подход к контролю выполнения технических мероприятий и учету результатов,
\item Сервис складского учета и управления запасами (сервис P4) — поддерживает контроль за движением запасных частей, обеспечением складов, формированием заявок на закупку,
\item Сервис аналитики и отчетности (сервис P5) — агрегирует данные для формирования аналитических материалов и отчетности для управления и надзорных органов.
\end{enumerate}

Язык Go [2], отличающийся высокой производительностью в обработке сетевых запросов и встроенной поддержкой конкурентности, представляет собой идеальное решение для сервисов с необходимостью обработки множества одновременных соединений. Для компонентов, где критически важны максимальная производительность и эффективное управление памятью, применяется C++ [3], обеспечивающий детальный контроль над ресурсами.

В качестве основного хранилища данных выбрана PostgreSQL – реляционная база данных, обеспечивающая согласованность, поддержку транзакций и возможности масштабирования. Redis [6] используется как промежуточный слой для кеширования и в роли брокера сообщений, что позволяет обеспечить быстрый доступ к данным и асинхронную коммуникацию между сервисами.

Связь между микросервисами реализована с использованием протоколов REST и gRPC в зависимости от требований к скорости и сложности взаимодействия. Асинхронная коммуникация поддерживается с помощью Redis Streams [7], позволяя уменьшить зависимость сервисов и повысить отказоустойчивость.

Для балансировки нагрузки и единой точки входа предусмотрен API Gateway, реализующий маршрутизацию, аутентификацию и базовые функции логирования и фильтрации запросов.

В архитектуру интегрированы современные механизмы безопасности: аутентификация и авторизация на базе OAuth 2.0 и JWT, контроль уровня доступа с разграничением прав по ролям, шифрование трафика и хранение конфиденциальных данных в соответствии с регламентами авиационной отрасли.

Для обеспечения качества и непрерывной поставки программного продукта используются системы контроля версий (Git), системы автоматической сборки и тестирования (CI/CD), контейнеризация приложений посредством Docker, оркестрация и управление жизненным циклом сервисов с помощью Kubernetes, централизованный мониторинг и логирование через Prometheus и ELK stack.

Выбранная архитектура и технологический стек создают прочную базу для эффективной разработки, поддержки и развития системы технического обслуживания воздушных судов, что обеспечит выполнение всех поставленных задач с высоким уровнем качества и надёжности.

\subsection{Описание технологического стека}

В процессе разработки распределённой микросервисной системы технического обслуживания воздушных судов особое внимание уделялось выбору технологического стека, способного обеспечить высокую производительность, надёжность, масштабируемость, а также простоту сопровождения и расширения функциональности. Ниже представлено детальное обоснование выбора ключевых технологий и инструментов.

\subsubsection{Обоснование выбора языков C++ и Go}

При выборе языков программирования для разработки микросервисов были учтены требования к скорости обработки, эффективному использованию ресурсов, удобству сопровождения и возможности поддержки современных парадигм разработки.

\begin{enumerate}
\item Высокая производительность. C++ традиционно обеспечивает низкоуровневый контроль над памятью и ресурсами процессора, что важно для сервисов, обрабатывающих большие потоки данных, выполняющих сложные вычислительные операции и требующих минимальных задержек. Особенно это актуально для сервисов планирования и управления складом, где операции с базой данных и алгоритмы требуют высокой производительности,

\item Современный язык для быстрого прототипирования и работы с конкурентностью. Go предоставляет простой и лаконичный синтаксис, встроенную поддержку параллельного программирования и эффективное управление сетевыми операциями. Это делает его оптимальным выбором для сервисов интеграции, управления исполнением работ и формирования аналитики, где важна скорость разработки и масштабируемость,

\item Модульность и независимость. Разделение функциональности между C++ и Go позволяет оптимально распределить нагрузку и повысить устойчивость системы. Такое решение снижает технологическую связанность и облегчает сопровождение каждого из сервисов,

\item Развитое экосистемное окружение. Оба языка обладают зрелыми сообществами и обширным набором библиотек, что снижает затраты на разработку и ускоряет переход от концепции к внедрению.
\end{enumerate}

\subsubsection{Использование PostgreSQL как основного хранилища данных}

В качестве основного хранилища выбран реляционный сервер баз данных PostgreSQL, который широко признан за высокую надежность и гибкость.

\begin{enumerate}
\item Надёжность и соответствие стандартам. PostgreSQL обеспечивает поддержку ACID-транзакций, что гарантирует целостность и корректность данных. Такой уровень надежности чрезвычайно важен для систем в авиационной сфере,

\item Гибкость и расширяемость. Возможность использовать собственные типы данных, расширять функционал с помощью пользовательских функций и хранить данные в формате JSON обеспечивает удобство адаптации базы под специфические требования,

\item Масштабируемость. Поддержка репликации и шардирования позволяет масштабировать базу под растущие объёмы и повышенную нагрузку,

\item Широкая интеграция. Наличие драйверов и поддержка различных протоколов облегчает взаимодействие с микросервисами на C++ и Go.
\end{enumerate}

\subsubsection{Роль Redis для кеширования и обмена сообщениями}

Для повышения производительности и обеспечения асинхронного обмена информацией между микросервисами применяется Redis [6] — высокопроизводительное in-memory-хранилище.

\begin{enumerate}
\item Кеширование часто запрашиваемых данных снижает нагрузку на основную базу данных и повышает скорость отклика системы,

\item Обеспечение асинхронного взаимодействия сервисов посредством Redis Streams позволяет строить устойчивую и отказоустойчивую архитектуру с минимальной связанностью компонентов,

\item Координация распределённых блокировок и управление сессиями пользователей реализуются через соответствующие механизмы Redis [6],

\item Наличие клиентских библиотек для C++ и Go упрощает интеграцию и разработку.
\end{enumerate}

\subsubsection{Инструменты разработки и контроля качества}

Для организации процессов разработки и поддержки системы применяются современные средства и методики:

\begin{enumerate}
\item Контейнеризация с использованием Docker позволяет создавать воспроизводимые и изолированные среды для разработки, тестирования и эксплуатации приложений,

\item Инструменты непрерывной интеграции и доставки (CI/CD) автоматизируют сборку, тестирование и деплой сервисов, что ускоряет процессы разработки и уменьшает риск человеческой ошибки,

\item Системы контроля версий (Git) обеспечивают совместную работу разработчиков и ведение истории изменений,

\item Тестирование реализуется на нескольких уровнях — модульное, интеграционное, нагрузочное и системное, что позволяет выявлять и устранять ошибки на ранних этапах,

\item Мониторинг и логирование с использованием современных инструментов (Prometheus, Grafana, ELK) обеспечивают своевременную диагностику сбоев и анализ производительности.
\end{enumerate}

Применение указанных инструментов и технологий формирует базу для эффективной разработки, эксплуатации и эволюции системы, удовлетворяющей высоким требованиям отрасли.

\subsection{Реализация ключевых микросервисов}

В рамках реализации системы технического обслуживания воздушных судов была выполнена детальная проработка и разработка ключевых микросервисов, обеспечивающих функциональность, описанную в архитектуре. Каждый из микросервисов выполняет строго определённые задачи и реализован с учётом особенностей предметной области, производительности и масштабируемости.

\subsubsection{Реализация сервиса управления справочниками (P1)}

Сервис управления справочными данными отвечает за интеграцию и поддержание актуальности информации, получаемой из различных внешних систем. В реализации данного сервиса использован язык Go, что обусловлено его преимуществами в области сетевого взаимодействия и обработки большого объёма входящих данных.

Для обмена данными с внешними системами реализованы REST API адаптеры, поддерживающие стандартные протоколы передачи информации в форматах JSON и XML. Сервис обеспечивает валидацию, нормализацию и дедупликацию данных перед записью в основное хранилище — базу данных PostgreSQL. При обнаружении изменённых данных генерация событий происходит через Redis Streams, что позволяет оповещать связанные микросервисы о необходимости актуализации их собственных данных.

Ряд функций сервиса включают синхронизацию справочников по расписанию, автоматический импорт данных и проверку целостности информации. Особое внимание уделено надежности и устойчивости: настроены механизмы повторных попыток в случае временных сбоев и логирование всех критичных операций.

\subsubsection{Реализация сервиса планирования (P2)}

Сервис планирования разработан на языке C++ с целью обеспечения высокой производительности, необходимой для сложного анализа и формирования оптимальных графиков технического обслуживания. В его основе лежат алгоритмы, учитывающие регламенты, состояние каждого воздушного судна, данные телеметрии и доступные ресурсы.

Планировщик реализован как автономный сервис с периодическим обновлением графиков и возможностью оперативного вмешательства при внеплановых событиях, инициируемых системой посредством событий из P1. Для взаимодействия с другими компонентами используется gRPC, обеспечивающий быструю и эффективно сериализуемую среду передачи данных.

В систему включена поддержка версионирования планов с сохранением истории изменений в базе данных, а также механизмы автоматического согласования планов с Центром управления полётами. Для обеспечения устойчивости и отказоустойчивости реализовано резервное копирование данных и возможность переключения на горячий резерв.

\subsubsection{Создание сервиса управления исполнением работ (P3)}

Данный сервис реализован на языке Go и предназначен для координации работы технического персонала на основе утверждённых планов. Он обеспечивает распределение заданий с учётом квалификации исполнителей, их текущей загруженности и особенностей определённых технических операций.

Интерфейс взаимодействия с пользователями построен на REST API и поддерживает как веб-клиентов, так и мобильные приложения. Важным элементом является система ввода результатов выполнения работ и ошибок, с последующим обновлением базы данных DS5 и генерацией уведомлений для служб контроля качества.

Дополнительно реализованы механизмы реализации офлайн-режима, позволяющие исполнителям работать в условиях ограниченного сетевого соединения, с последующей синхронизацией данных при восстановлении связи. Центральное хранилище состоит из PostgreSQL и Redis, обеспечивая высокую скорость доступа и стабильность.

\subsubsection{Реализация сервиса управления складом и запасами (P4)}

Для контроля складских запасов применён язык C++ с использованием специализированных библиотек для обработки больших массивов данных. Сервис взаимодействует с внешними складскими системами и реализует функции учёта товара, резервирования материалов и формирования заявок на закупку.

Для повышения производительности реализованы многопоточные алгоритмы синхронизации данных и обработки запросов. Взаимодействие с другими микросервисами построено на базе распределённых событий, организованных через Redis Streams.

Сервис оснащён интерфейсами для управленческого доступа и аудита изменений, а также имеет встроенный модуль безопасности, гарантирующий соответствие отраслевым стандартам по разграничению прав.

\subsubsection{Разработка модуля аналитики и отчетности (P5)}

Модуль аналитики и отчетности создан с применением языка Go с целью эффективной обработки больших объёмов данных, поступающих из всех подсистем. Он реализует функции формирования разнообразных отчётов, таблиц, графиков и статистических сводок, необходимых для руководства, надзорных органов и планово-логистических служб.

Для подготовки отчётов задействованы механизмы агрегации данных из PostgreSQL и кеша Redis, что позволяет обеспечить быстрый ответ на пользовательские запросы. Модуль поддерживает генерацию документов в формате PDF, Excel и предоставляет данные через API для интеграции с внешними системами бизнес-аналитики.

В задачи модуля входит также прогнозирование технических нагрузок, анализ выполненных работ и выявление трендов, способствующих принятию управленческих решений. Внедрена система разграничения доступа для обеспечения конфиденциальности информации.

Данные ключевые микросервисы совместно обеспечивают полное покрытие бизнес-процессов технического обслуживания воздушных судов, а их архитектура и технологическая реализация поддерживают требования по надёжности, масштабируемости и дальнейшему развитию системы.

\subsection{Организация хранения и управления данными}

Организация надежного и эффективного хранения данных является одной из ключевых задач при разработке распределённой микросервисной системы технического обслуживания воздушных судов. Выбранные решения обеспечивают целостность, быстродействие и масштабируемость информационного слоя системы, что критически важно для успешного функционирования в условиях высоких нагрузок и строгих требований отрасли.

\subsubsection{Моделирование базы данных PostgreSQL: структуры и связи}

Для хранения основных данных системы используется реляционная база PostgreSQL, обладающая широкими возможностями масштабирования и высокой отказоустойчивостью. При проектировании базы данных выполнено создание унифицированной схемы, включающей следующие основные компоненты и связи между ними:

\begin{enumerate}
\item Справочники и справочные таблицы, включающие информацию о воздушных судах, используемых моделях, конфигурациях и технических параметрах. Каждая сущность имеет уникальный идентификатор, обеспечивающий целостность данных,

\item Данные о персонале, структурированно хранящие сведения о техниках, супервайзерах, их квалификациях, ролях и полномочиях. Взаимоотношения между сотрудниками и выполняемыми работами реализованы через внешние ключи,

\item Планы и регламенты технического обслуживания, содержащие классификации видов ТО, периодичности и графики. Модель проектирования учитывает возможность версионирования планов и учёта их исторических изменений,

\item Записи о выполненных работах, включающие дату, участников, использованные запчасти, выявленные дефекты и результаты контроля качества. Эти данные связаны с соответствующими записями в справочниках и планах для обеспечения полной прослеживаемости,

\item Управление складскими запасами, где хранятся данные о текущих остатках, движениях и заказах на запчасти, связанное с карточками запасов и внутренними заявками.
\end{enumerate}

При моделировании активно использованы индексы, ограничители целостности и триггеры для автоматизации контроля и обеспечения высокой производительности запросов. Специализированные представления и функции оптимизируют выборки и агрегирование данных, особенно при формировании отчетности.

\subsubsection{Настройка кеширования и сессий в Redis}

Для повышения скорости обработки запросов и разгрузки основной базы данных в системе внедрён кеширующий слой на основе Redis. Основные аспекты использования Redis включают:

\begin{enumerate}
\item Кеширование часто запрашиваемых данных, таких как справочники, текущие статусы работ и доступные складские остатки, что значительно снижает задержки отклика системы,

\item Управление сессиями пользователей и хранение временной информации, необходимой для работы распределённых сервисов, включая управление блокировками и семафорами,

\item Организация очередей сообщений и событий, применяемых для асинхронного межсервисного взаимодействия через структуру Redis Streams, что повышает отказоустойчивость и масштабируемость,

\item Внедрены механизмы мониторинга состояния кеша и автоматического обновления данных, позволяющие избегать устаревшей информации и поддерживать синхронизацию с основной СУБД.
\end{enumerate}

\subsubsection{Обеспечение миграций и резервного копирования}

Для управления версиями базы данных, контролируемых изменений структуры и данных используемых сущностей, внедрена система миграций, позволяющая:

\begin{enumerate}
\item Автоматизировать обновления схемы базы данных без простоев системы,

\item Обеспечить безопасность миграций с возможностью отката на предыдущие версии в случае ошибок,

\item Сохранять историю изменений для аудита и упрощения сопровождения.
\end{enumerate}

Резервное копирование реализовано с использованием стандартных механизмов PostgreSQL:

\begin{enumerate}
\item Регулярное полное и инкрементальное резервное копирование данных,

\item Настройка репликации для обеспечения высокой доступности и оперативного восстановления после сбоев,

\item Использование внешних систем хранения и автоматизация процедур восстановления данных.
\end{enumerate}

Эти меры гарантируют сохранность критической информации и минимизацию рисков потери данных, что является необходимым условием для соответствия высоким стандартам авиационной индустрии.

Таким образом, комплексная организация хранения и управления данными включает продуманное проектирование реляционной базы, эффективное применение механизма кеширования и современных технологий миграции и резервного копирования, обеспечивая устойчивость, производительность и безопасность системы технического обслуживания воздушных судов.

\subsection{Межсервисное взаимодействие}

Эффективная координация и обмен данными между микросервисами является основополагающим условием построения масштабируемой и надёжной системы технического обслуживания воздушных судов. Рассмотренные в данном разделе решения обеспечивают целостность, производительность и отказоустойчивость взаимодействий в распределённой среде.

\subsubsection{Используемые протоколы и форматы}

В качестве основных протоколов для синхронного обмена информацией между сервисами используются REST[5] и gRPC[6].  

REST-интерфейсы поддерживают стандартные возможности HTTP с передачей данных в формате JSON, что обеспечивает простоту интеграции и широкую совместимость с внешними системами и пользовательским интерфейсом.  

gRPC применяется для взаимодействия, требующего высокой пропускной способности и эффективной сериализации сообщений. Используемый протокол основан на HTTP/2 и протоколе буферизации protobuf, что существенно ускоряет обмен данными и снижает нагрузку на сеть.

Выбор формата JSON для REST API обусловлен его широкой распространённостью и удобством в отладке, а protobuf — высокой компактностью и скоростью обработки — оптимален для межсервисного обмена внутри кластера.

\subsubsection{Обработка транзакций и согласованность данных}

В распределённой системе обеспечение согласованности данных представляет значительную сложность, особенно при наличии асинхронного взаимодействия между сервисами.  

Для критичных операций применяется локальное управление транзакциями внутри каждого микросервиса с использованием возможностей СУБД PostgreSQL, что гарантирует ACID-свойства на уровне отдельного сервиса.  

Межсервисные процессы согласования реализуются на основе паттернов саги (SAGA), при которых комплексные бизнес-транзакции разбиваются на последовательность локальных операций с возможностью компенсаций в случае неудач. Это позволяет поддерживать консистентное состояние данных без блокировок и снижать риски взаимных зависимостей между сервисами.

Дополнительно внедрена логика оповещения и повторных попыток при сбоях коммуникации, что повышает надёжность работы распределённой системы.

\subsubsection{Асинхронное взаимодействие и использование Redis Streams}

Для асинхронной коммуникации применяется механизм потоков сообщений Redis Streams. Данный инструмент предоставляет:

\begin{enumerate}
\item гарантию доставки сообщений между микросервисами с возможностью считывать их несколькими потребителями,

\item упорядоченный транзакционный поток событий, что упрощает интеграцию и консистентную обработку,

\item максимальную производительность и низкую задержку, обеспечивая масштабируемость и гибкость интеграции.
\end{enumerate}

Асинхронные сообщения используются для оповещения об изменениях данных, триггерах в бизнес-логике, обновлении кеша, а также для передачи производственных событий, что существенно разгружает синхронные каналы и уменьшает связанность микросервисов.

\subsection{Тестирование и обеспечение качества}

Обеспечение высокого качества программного продукта является важнейшей задачей при разработке системы технического обслуживания воздушных судов. Для достижения требуемого уровня надёжности, производительности и устойчивости была построена комплексная система тестирования, включающая различные типы испытаний и подготовку необходимых тестовых данных.

В настоящем разделе описаны этапы тестирования, применяемые методики и результаты, а также организация контроля качества в процессе разработки.

В ходе проектирования и реализации системы выполнялись следующие виды испытаний:

\begin{enumerate}[label=\alph*)]
\item модульное тестирование, направленное на проверку корректности отдельных компонентов и бизнес-логики в изоляции,

\item интеграционное тестирование, обеспечивающее проверку взаимодействия между микросервисами и с внешними системами,

\item нагрузочное тестирование, позволяющее оценить производительность системы при реальных и экстремальных нагрузках.
\end{enumerate}

Для поддержания высокого качества программного обеспечения все тесты интегрированы в конвейер непрерывной интеграции, что позволяет автоматически запускать проверки при каждом обновлении исходного кода. Такой подход обеспечивает своевременное выявление ошибок и снижает риск возникновения регрессий.

Особое внимание уделялось генерации и подготовке тестовых данных, максимально приближенных к реальным условиям эксплуатации системы:

\begin{enumerate}[label=\alph*)]
\item формирование моделей воздушных судов с разнообразными конфигурациями и состояниями агрегатов,

\item имитация динамики наработки оборудования на основе статистических и технических норм,

\item создание профилей технического состава с различной квалификацией и распределением ролей,

\item моделирование плановых и внеплановых мероприятий технического обслуживания в соответствии с регламентами,

\item симуляция работы складских запасов и процесса перемещения запчастей.
\end{enumerate}

Для реализации имитации был реализован специализированный генератор данных, позволяющий создавать масштабируемые и реалистичные объёмы информации, что существенно повысило качество и адекватность нагрузочного тестирования.

Для проведения нагрузочных испытаний использовались современные инструменты, такие как Apache JMeter и Locust, что позволило гибко управлять сценариями нагрузки и полноценно собирать статистические показатели производительности.

В процессе тестирования оценивались пропускная способность системы, время отклика на запросы, стабильность работы при максимальных нагрузках, а также устойчивость при длительной эксплуатации. Полученные результаты использовались для оптимизации архитектуры, настройки инфраструктуры и повышения производительности.

В целях контроля качества на всех этапах разработки внедрены дополнительные методики:

\begin{enumerate}[label=\alph*)]
\item регулярный статический анализ кода, выявляющий ошибки, нарушения стиля и потенциальные уязвимости,

\item использование практики разработки через тестирование (TDD), позволяющей создавать более надёжный и тестируемый код,

\item организация коллективного ревью кода для повышения качества решений,

\item мониторинг состояния сервисов с помощью Prometheus и визуализация данных через Grafana,

\item централизованное логирование с использованием стека ELK (Elasticsearch, Logstash, Kibana) для оперативного анализа инцидентов.
\end{enumerate}

Суммарно интеграция всех перечисленных методик и средств обеспечила создание высококачественного, надёжного и производительного программного комплекса, полностью отвечающего требованиям безопасности и регламентации авиационной отрасли.


\newpage
\newpage
%%%%%%%% РАЗДЕЛ 4 %%%%%%%%
\section{РАЗВЁРТЫВАНИЕ И ЭКСПЛУАТАЦИЯ}

\subsection{Контейнеризация и оркестрация (Docker, Kubernetes)}

Для обеспечения переносимости, масштабируемости и унифицированного управления микросервисами система использует контейнеризацию на базе Docker. Контейнеры создают стандартизированное и изолированное окружение, что упрощает развертывание и эксплуатацию приложений.

На рисунке 3 представлена общая схема контейнеризации микросервисов системы технического обслуживания воздушных судов.

% Место для рисунка 3
% \begin{figure}[h]
% \centering
% \includegraphics[width=0.9\textwidth]{docker-schema.png}
% \caption{Схема контейнеризации микросервисов}
% \end{figure}

Каждый из микросервисов системы упаковывается в отдельный контейнер со всеми необходимыми зависимостями, библиотеками и настройками. Такой подход обеспечивает:

\begin{enumerate}
\item изоляцию микросервисов и устранение конфликтов между зависимостями,
\item возможность быстрого развертывания и перезапуска отдельных компонентов системы,
\item снижение затрат на подготовку инфраструктуры и настройку среды выполнения,
\item упрощение процесса тестирования и вывода новых версий в производственную среду.
\end{enumerate}

Оркестрация контейнеров осуществляется с помощью Kubernetes, что позволяет автоматизировать управление жизненным циклом сервисов, включая масштабирование, балансировку нагрузки, обновления без остановки работы и самовосстановление после сбоев.

Основные конфигурации Kubernetes, разработанные для системы технического обслуживания, включают:

\begin{enumerate}
\item Deployment-манифесты для каждого микросервиса с настройками репликации и стратегиями обновления,
\item Service-объекты для обеспечения стабильных точек доступа к сервисам внутри кластера,
\item Ingress-контроллер для управления входящим трафиком и маршрутизации запросов от внешних пользователей,
\item ConfigMap и Secret для хранения конфигурационных данных и чувствительной информации,
\item PersistentVolumeClaim для постоянного хранения данных, переживающих перезапуск контейнеров,
\item HorizontalPodAutoscaler для автоматического масштабирования на основе нагрузки.
\end{enumerate}

Для непрерывной поставки и развертывания новых версий системы создан конвейер CI/CD на базе GitLab CI, который обеспечивает автоматическое тестирование, сборку контейнеров и их развертывание в тестовое или производственное окружение. Процесс развертывания следует стратегии Rolling Update, обеспечивающей обновление сервисов без простоя системы.

Настроены механизмы мониторинга и логирования, интегрированные с Kubernetes, для своевременного обнаружения проблем и поддержания высокой доступности системы. Компоненты инфраструктуры развертывания представлены в таблице 3.

\begin{table}[h]
\caption{Компоненты инфраструктуры контейнеризации и оркестрации}
\begin{tabular}{|p{4cm}|p{10cm}|}
\hline
\textbf{Компонент} & \textbf{Назначение} \\
\hline
Docker & Контейнеризация микросервисов и их зависимостей \\
\hline
Kubernetes & Оркестрация и управление жизненным циклом контейнеров \\
\hline
Helm & Управление пакетами Kubernetes и шаблонизация манифестов \\
\hline
Harbor & Приватный реестр Docker-образов с функциями сканирования уязвимостей \\
\hline
GitLab CI/CD & Автоматизация сборки, тестирования и развертывания \\
\hline
\end{tabular}
\end{table}

Таким образом, использование технологий контейнеризации и оркестрации обеспечивает гибкость, надежность и масштабируемость системы технического обслуживания воздушных судов на уровне инфраструктуры.

\subsection{Мониторинг, логирование и алёртинг}

Для обеспечения надежной и эффективной работы распределенной системы технического обслуживания воздушных судов разработана комплексная стратегия мониторинга, логирования и оповещения. Эта стратегия позволяет своевременно выявлять и устранять проблемы, оптимизировать использование ресурсов и анализировать поведение системы.

Система мониторинга основана на Prometheus, который собирает метрики работы сервисов и инфраструктуры. Основные группы метрик включают:

\begin{enumerate}
\item системные метрики (загрузка CPU, использование памяти, дисковое пространство),
\item метрики сети (задержки, пропускная способность, количество соединений),
\item метрики приложений (время ответа API, количество запросов, частота ошибок),
\item бизнес-метрики (количество обработанных планов, время выполнения технических работ).
\end{enumerate}

Все микросервисы системы реализуют стандартизированные эндпоинты для предоставления метрик в формате, понятном Prometheus. Для сбора метрик из компонентов инфраструктуры используются экспортеры, такие как Node Exporter для системных метрик и PostgreSQL Exporter для метрик баз данных.

Для визуализации данных используется Grafana, обеспечивающая удобные дашборды и возможность детального анализа состояния системы. Разработаны следующие основные дашборды:

\begin{enumerate}
\item общий дашборд состояния системы с ключевыми показателями,
\item детальные дашборды для каждого микросервиса,
\item дашборд производительности баз данных и кеша,
\item дашборд сетевого взаимодействия между компонентами.
\end{enumerate}

Централизованное логирование реализовано с помощью стека ELK (Elasticsearch, Logstash, Kibana), что облегчает агрегирование, поиск и анализ логов всех компонентов. Logs Collector на базе Filebeat собирает логи из всех контейнеров и отправляет их в Logstash для фильтрации и структурирования. Далее обработанные логи сохраняются в Elasticsearch и становятся доступны для поиска и анализа через Kibana.

Для обеспечения единообразия и структурирования логов во всех микросервисах принят стандарт форматирования, включающий:

\begin{enumerate}
\item уровень логирования (INFO, WARN, ERROR, DEBUG),
\item идентификатор запроса для трассировки взаимодействий между сервисами,
\item метку времени в унифицированном формате,
\item контекст события (сервис, компонент, функция),
\item детализированное сообщение.
\end{enumerate}

Настроена система алёртинга на базе Alertmanager, отправляющая оперативные уведомления в случае возникновения критических событий или отклонений от нормы, что позволяет минимизировать время реакции на инциденты. Уведомления настроены по нескольким каналам связи, включая:

\begin{enumerate}
\item электронную почту для некритичных предупреждений,
\item системы мгновенных сообщений (Slack, Telegram) для оперативных уведомлений,
\item СМС-оповещения для критических инцидентов,
\item интеграцию с системами управления инцидентами (Jira, ServiceNow).
\end{enumerate}

Правила оповещения структурированы по уровням серьезности и сгруппированы для предотвращения шквала уведомлений при взаимосвязанных проблемах. Настроены механизмы дедупликации и подавления повторяющихся оповещений.

Для более глубокой диагностики проблем и понимания взаимодействия между сервисами внедрена система распределенной трассировки на базе Jaeger. Это позволяет отслеживать путь запроса через все микросервисы системы, выявлять узкие места и оптимизировать взаимодействие компонентов.

Комбинация этих инструментов обеспечивает всесторонний контроль над состоянием системы технического обслуживания воздушных судов, позволяет поддерживать высокий уровень доступности и производительности, а также быстро реагировать на возникающие проблемы.

\subsection{Особенности безопасности и управления доступом}

Обеспечение информационной безопасности является критическим аспектом системы технического обслуживания воздушных судов, учитывая чувствительность данных и потенциальные риски, связанные с авиационной отраслью. В данном разделе описаны основные меры и механизмы, реализованные для защиты системы, данных и коммуникаций.

Внедрена аутентификация и авторизация на базе протоколов OAuth 2.0 и JWT, обеспечивающих защищённый и масштабируемый механизм контроля доступа. Данный подход позволяет реализовать:

\begin{enumerate}
\item централизованное управление учетными данными пользователей,
\item поддержку единого входа (SSO) при интеграции с корпоративными системами идентификации,
\item делегирование полномочий без передачи учетных данных,
\item ограниченный по времени доступ с использованием токенов с истекающим сроком действия.
\end{enumerate}

Реализовано разграничение прав пользователей и сервисов с назначением ролей, что гарантирует доступ только к необходимым функциям и данным. Система ролевого доступа (RBAC) включает следующие основные категории пользователей:

\begin{enumerate}
\item администраторы системы с полным доступом к управлению и настройке,
\item руководители технических служб с доступом к планированию и отчетности,
\item технические специалисты с доступом к данным о заданиях и отчетам о выполнении,
\item специалисты по контролю качества с доступом к данным проверок и инспекций,
\item персонал склада с доступом к учету и движению запчастей,
\item интеграционные сервисы с ограниченным программным доступом к определенным API.
\end{enumerate}

Все коммуникации между компонентами системы и внешними пользователями шифруются с использованием TLS, обеспечивая конфиденциальность и защиту от перехвата данных. Настроено автоматическое обновление сертификатов с использованием сервиса Cert-Manager в Kubernetes, интегрированного с Let's Encrypt.

Для защиты от распространенных атак веб-приложений внедрен Web Application Firewall (WAF), который обеспечивает:

\begin{enumerate}
\item защиту от SQL-инъекций,
\item защиту от XSS (Cross-Site Scripting),
\item предотвращение CSRF-атак (Cross-Site Request Forgery),
\item ограничение частоты запросов (rate limiting) для защиты от DDoS-атак.
\end{enumerate}

Система хранения чувствительных данных (паролей, ключей API, сертификатов) реализована с использованием HashiCorp Vault, который обеспечивает безопасное хранение, управление доступом и ротацию секретов. Все конфиденциальные данные хранятся в зашифрованном виде с использованием современных криптографических алгоритмов.

Разработана и внедрена стратегия обеспечения соответствия требованиям законодательства о защите персональных данных, включая:

\begin{enumerate}
\item идентификацию и классификацию персональных данных в системе,
\item механизмы псевдонимизации и анонимизации данных, где это возможно,
\item политики хранения и удаления данных в соответствии с требованиями,
\item процедуры обработки запросов субъектов персональных данных.
\end{enumerate}

Проводится регулярный аудит безопасности, включающий:

\begin{enumerate}
\item сканирование кода на наличие уязвимостей,
\item анализ зависимостей и используемых компонентов,
\item проверку конфигураций на соответствие лучшим практикам безопасности,
\item периодическое тестирование на проникновение (penetration testing).
\end{enumerate}

Для эффективного реагирования на инциденты безопасности разработаны политики и процедуры, определяющие:

\begin{enumerate}
\item порядок выявления и классификации инцидентов,
\item ответственность и роли участников процесса реагирования,
\item шаги по локализации, расследованию и устранению последствий,
\item механизмы коммуникации и эскалации.
\end{enumerate}

Комплексный подход к обеспечению безопасности гарантирует защиту системы технического обслуживания воздушных судов от различных угроз и уязвимостей, обеспечивая целостность данных и непрерывность критически важных бизнес-процессов авиационных предприятий.

\newpage 

\section{ИТОГИ РАЗРАБОТКИ И ПЕРСПЕКТИВЫ РАЗВИТИЯ}

\subsection{Результаты, достижения, выявленные проблемы}

В рамках проекта создана масштабируемая микросервисная система, обеспечивающая полноценное автоматизированное управление техническим обслуживанием воздушных судов. Выполнение поставленных целей и задач позволило достичь значимых результатов, которые можно оценить как с технической, так и с организационной точек зрения.

Ключевыми достижениями проекта являются:

\begin{enumerate}
\item Разработка и внедрение целостной архитектуры системы на основе микросервисного подхода, обеспечивающей четкое разделение ответственности компонентов и возможность их независимой эволюции. Архитектура поддерживает горизонтальное масштабирование каждого сервиса в зависимости от нагрузки и специфики эксплуатации,

\item Создание пяти ключевых подсистем (P1-P5), полностью покрывающих потребности авиационных предприятий в автоматизации процессов технического обслуживания – от управления справочными данными до формирования аналитики и отчетности,

\item Реализация гибкой системы хранения данных, включающей реляционные базы, кеширование и механизмы обмена сообщениями, что обеспечивает оптимальную производительность при сохранении целостности информации,

\item Разработка интеграционных механизмов, позволяющих системе эффективно взаимодействовать с существующими корпоративными информационными системами авиапредприятий, такими как ERP, HR и системы планирования полетов,

\item Внедрение комплексных мер безопасности, обеспечивающих защиту чувствительных данных и соответствие нормативным требованиям авиационной отрасли,

\item Создание инфраструктуры для автоматизированного тестирования, развертывания и мониторинга, гарантирующей высокое качество и стабильность работы системы.
\end{enumerate}

Технические характеристики разработанной системы подтверждают ее соответствие поставленным требованиям:

\begin{enumerate}
\item Система способна обрабатывать до 1000 запросов в секунду при пиковых нагрузках,
\item Время отклика на стандартные запросы не превышает 200 мс при нормальных условиях эксплуатации,
\item Достигнута доступность системы на уровне 99,95\% в соответствии с SLA,
\item Горизонтальное масштабирование позволяет увеличивать производительность линейно с добавлением вычислительных ресурсов,
\item Реализована поддержка одновременной работы до 500 пользователей без деградации производительности.
\end{enumerate}

В процессе разработки и тестирования системы были выявлены определенные проблемы и ограничения, которые требуют дальнейшего внимания:

\begin{enumerate}
\item Сложность обеспечения согласованности данных в распределенной микросервисной архитектуре, особенно при высоких нагрузках и параллельной обработке запросов,
\item Необходимость дополнительной оптимизации алгоритмов планирования для более эффективного распределения ресурсов при сложных сценариях технического обслуживания,
\item Ограничения существующих интеграционных протоколов при взаимодействии с устаревшими системами некоторых авиапредприятий,
\item Высокие требования к квалификации персонала для эффективного использования всех возможностей системы,
\item Потребность в дальнейшем развитии мобильных интерфейсов для работы технического персонала в полевых условиях.
\end{enumerate}

Несмотря на выявленные ограничения, разработанная система технического обслуживания воздушных судов успешно проходит опытную эксплуатацию и подтверждает свою эффективность в реальных условиях. Пользователи отмечают интуитивность интерфейса, скорость работы и удобство получения необходимой информации для принятия решений.

Экономическая эффективность внедрения системы оценивается в снижении времени простоя воздушных судов на 15-20\%, сокращении затрат на управление запасными частями на 10-12\% и повышении производительности технического персонала на 25\% за счет оптимизации планирования и распределения работ.

\subsection{Планы по развитию системы и расширению функционала}

На основе результатов разработки и обратной связи от пользователей определены перспективные направления развития системы технического обслуживания воздушных судов, которые будут реализованы в последующих версиях.

Ключевыми направлениями развития являются:

\begin{enumerate}
\item Интеграция дополнительных источников телеметрических данных для повышения точности прогнозирования состояния оборудования. Планируется расширить спектр поддерживаемых интерфейсов для получения данных с бортовых систем различных типов воздушных судов, включая ARINC 429, AFDX и нестандартные проприетарные протоколы. Это позволит получать более детальную информацию о состоянии критически важных систем и компонентов в режиме реального времени,

\item Разработка и внедрение модулей предиктивной аналитики с использованием машинного обучения для своевременного выявления потенциальных неисправностей. Будут реализованы алгоритмы на основе технологий искусственного интеллекта, способные анализировать исторические данные о неисправностях, телеметрию и условия эксплуатации для прогнозирования возможных отказов до их фактического возникновения. Это критически важно для перехода от планового технического обслуживания к превентивному, основанному на фактическом состоянии оборудования,

\item Расширение возможностей складского учёта с учётом динамически изменяющихся требований и особенностей логистики. Планируется разработка модулей оптимизации закупок и распределения запасных частей между различными техническими базами с использованием методов операционного исследования. Также будет реализована поддержка альтернативных и взаимозаменяемых запчастей, что особенно актуально в условиях возможных логистических ограничений,

\item Создание мобильных и голосовых интерфейсов для упрощения взаимодействия технического персонала с системой. Разработка кроссплатформенных мобильных приложений для iOS и Android с поддержкой работы в офлайн-режиме и последующей синхронизацией позволит техническим специалистам эффективно использовать систему непосредственно на местах проведения работ. Внедрение голосового управления с использованием технологий распознавания речи даст возможность работать с системой в условиях, когда руки заняты выполнением технических операций,

\item Усиление мер информационной безопасности, включая внедрение средств обнаружения вторжений и контроля целостности. Планируется интеграция с современными решениями по обнаружению и предотвращению вторжений (IDS/IPS), а также внедрение механизмов постоянного мониторинга целостности критически важных данных и компонентов системы. Дополнительно будут реализованы процедуры регулярного аудита безопасности и тестирования на проникновение,

\item Расширение интеграционных возможностей с системами финансового учета и ERP для построения цельной экосистемы управления ресурсами предприятия. Это позволит автоматизировать финансовое планирование и учет затрат на техническое обслуживание, оптимизировать бюджетирование и контроль расходов,

\item Разработка модулей поддержки международных стандартов и требований авиационных властей различных стран, что расширит географию применения системы и упростит процессы сертификации.
\end{enumerate}

Для обеспечения эффективной эволюции системы разработана дорожная карта развития, разделенная на краткосрочные (6-12 месяцев), среднесрочные (1-2 года) и долгосрочные (3-5 лет) перспективы. Это позволяет планомерно распределять ресурсы и обеспечивать постепенное наращивание функциональности в соответствии с приоритетами пользователей и требованиями рынка.

В краткосрочной перспективе фокус будет сделан на оптимизации существующих алгоритмов, улучшении пользовательского интерфейса и расширении интеграционных возможностей. В среднесрочной перспективе планируется внедрение модулей предиктивной аналитики и мобильных интерфейсов. Долгосрочные планы включают развитие возможностей искусственного интеллекта и адаптацию системы к новым стандартам и требованиям авиационной отрасли.

Особое внимание будет уделено обеспечению обратной совместимости и плавной миграции данных при обновлении системы, чтобы минимизировать влияние изменений на операционную деятельность авиапредприятий.

Проведение регулярно обновляемой оценки эффективности системы и адаптация под новые требования отрасли останется неотъемлемой частью процесса развития, обеспечивая актуальность и конкурентоспособность разработанного решения на рынке систем технического обслуживания воздушных судов.

\newpage

%%%%%%%% ЗАКЛЮЧЕНИЕ %%%%%%%%
\headersection{ЗАКЛЮЧЕНИЕ}

В ходе выполнения данной работы разработана распределенная микросервисная система планирования технического обслуживания воздушных судов, отвечающая современным требованиям авиационной отрасли. Проведенное исследование подтверждает актуальность проблемы создания отечественного программного обеспечения для данной предметной области, учитывающего национальные регламенты и обладающего гибкостью для интеграции с существующими информационными системами авиакомпаний.

Основные результаты работы заключаются в следующем:

1. Проведен комплексный анализ существующих зарубежных систем автоматизации технического обслуживания воздушных судов, таких как Veryon, AMOS и Skywise, выявлены их сильные стороны и ограничения для применения в российских условиях.

2. Разработана модульная архитектура системы, основанная на микросервисном подходе, обеспечивающая высокую масштабируемость, надежность и гибкость при внедрении в различных авиакомпаниях и сервисных организациях.

3. Реализованы ключевые компоненты системы: подсистема управления справочными данными, подсистема планирования обслуживания, подсистема контроля исполнения работ, подсистема управления складом и аналитический модуль для формирования отчетности.

4. Спроектирована эффективная структура хранения данных, включающая реляционную базу PostgreSQL и кеширующий слой Redis, что обеспечивает высокую производительность при значительных нагрузках.

5. Разработана система межсервисного взаимодействия с использованием современных протоколов, обеспечивающая устойчивость и надежность обмена данными между компонентами.

6. Организована система тестирования и обеспечения качества, гарантирующая соответствие программного продукта строгим требованиям авиационной отрасли к безопасности и надежности.

Поставленные задачи выполнены в полном объеме. Разработанное решение отличается от существующих аналогов следующими преимуществами:

1. Адаптированность к требованиям российского законодательства и регламентам обслуживания воздушных судов.

2. Возможность гибкой интеграции с существующими на предприятиях информационными системами.

3. Оптимизация затрат на внедрение и сопровождение по сравнению с зарубежными аналогами.

4. Высокая масштабируемость и производительность, достигнутая за счет микросервисной архитектуры и правильного выбора технологического стека.

Экономическая эффективность внедрения разработанной системы заключается в сокращении времени простоя воздушных судов, оптимизации использования ресурсов технического персонала и запасных частей, а также в снижении зависимости от зарубежных поставщиков программного обеспечения.

Перспективы дальнейшего развития системы включают интеграцию средств машинного обучения для предиктивной аналитики состояния воздушных судов, расширение мобильных интерфейсов для полевого персонала и совершенствование средств визуализации данных для управленческого персонала.

Таким образом, разработанная система представляет собой современное, конкурентоспособное решение, способное эффективно удовлетворить потребности авиационной отрасли в области автоматизации процессов технического обслуживания воздушных судов и обеспечить технологическую независимость от зарубежных решений.
\clearpage

%%%%%%%% СПИСОК ИСПОЛЬЗОВАННЫХ ИСТОЧНИКОВ %%%%%%%%
\headersection{СПИСОК ИСПОЛЬЗОВАННЫХ ИСТОЧНИКОВ}

{\sloppy  % Начало режима sloppy для лучшего переноса строк
\begin{enumerate}
\item Российская Федерация. Федеральные авиационные правила. Требования к юридическим лицам, индивидуальным предпринимателям, осуществляющим техническое обслуживание гражданских воздушных судов : ФАП-285 : утверждены Приказом Минтранса России от 25.09.2015 № 285 : текст с изменениями и дополнениями на 27.10.2023 года. -- Текст : непосредственный // Бюллетень нормативных актов федеральных органов исполнительной власти. -- 2023. -- № 47.

\item Types of Aviation Maintenance Checks: A, B, C, and D // NSL Aerospace [Электронный ресурс]. \url{https://nslaerospace.com/types-of-aviation-maintenance-checks/} (дата обращения: 08.05.2025).

\item The Go Programming Language Documentation // Golang.org [Электронный ресурс]. \url{https://golang.org/doc/} (дата обращения: 15.05.2023).

\item Standard C++ Foundation [Электронный ресурс]. \url{https://isocpp.org/} (дата обращения: 08.05.2025).

\item Newman S. Building Microservices: Designing Fine-Grained Systems. -- O'Reilly Media, 2021. -- 616 с.

\item Fielding R.T. Architectural Styles and the Design of Network-based Software Architectures: дис. -- University of California, Irvine, 2000. -- 162 с.

\item PostgreSQL Documentation, Version 14 [Электронный ресурс]. \url{https://www.postgresql.org/docs/14/index.html} (дата обращения: 08.05.2025).

\item Swiss Aviation Software (Swiss-AS) [Электронный ресурс]. \url{https://www.swiss-as.com/} (дата обращения: 08.05.2025).

\item Lufthansa German Airlines and Lufthansa Cargo live with AMOS // Swiss Aviation Software [Электронный ресурс]. \url{https://amosng.swiss-as.com/news-events/news/lufthansa-german-airlines-and-lufthansa-cargo-live-amos} (дата обращения: 08.05.2025).

\item Boeing Insight Accelerator for Maintenance Optimization [Электронный ресурс]. \url{https://services.boeing.com/maintenance-engineering/maintenance-optimization/insight-accelerator} (дата обращения: 08.05.2025).
\end{enumerate}
}  % Конец режима sloppy
\clearpage


%%%%%%%% ПРИЛОЖЕНИЕ А %%%%%%%%
\headersection{ПРИЛОЖЕНИЕ А}

\subsection*{Схема взаимодействия микросервисов системы}

В данном приложении представлена детальная схема взаимодействия микросервисов разработанной системы технического обслуживания воздушных судов. Схема демонстрирует информационные потоки между основными компонентами системы, внешними источниками данных и пользовательскими интерфейсами.

% Тут должен быть рисунок
% \begin{figure}[h]
% \centering
% \includegraphics[width=\textwidth]{schema.png}
% \caption{Схема взаимодействия микросервисов системы}
% \end{figure}
\clearpage

%%%%%%%% ПРИЛОЖЕНИЕ Б %%%%%%%%
\headersection{ПРИЛОЖЕНИЕ Б}

\subsection*{Примеры интерфейсов пользователя}

В данном приложении представлены макеты основных пользовательских интерфейсов системы:

\begin{enumerate}
\item Панель планирования технического обслуживания с календарным представлением и ресурсной загрузкой.

\item Интерфейс контроля выполнения работ с отображением текущего статуса задач.

\item Аналитический дашборд с ключевыми показателями эффективности технического обслуживания.

\item Форма учета и списания запасных частей при выполнении технических работ.

\item Интерфейс формирования и просмотра отчетности для контролирующих органов.
\end{enumerate}

\subsection*{Описание API для интеграции с внешними системами}

В данном разделе приложения представлено описание программных интерфейсов (API), предназначенных для интеграции с внешними информационными системами, включая системы планирования полетов, кадрового учета и управления запасами. Для каждого интерфейса указаны поддерживаемые методы, форматы данных и примеры запросов и ответов.

\end{document}
